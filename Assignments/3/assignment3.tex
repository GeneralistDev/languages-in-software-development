\documentclass[11pt, numbers=endperiod, parskip=half]{scrartcl}

\usepackage{amsmath}
\usepackage{color}
\usepackage{semantic}
\usepackage{minted}

\title{Assignment 3}
\subtitle{COS30023 - Languages in Software Development}
\author{Daniel Parker - 971328X}

\date{\today}

\begin{document}
\maketitle

\section{Problem 1}
\subsection{String specification}
\begin{align*}
S ::=& \ \ \ \epsilon \\
&|\ \ aS
\end{align*}
\subsection{String Length Specification}

\begin{align*}
length(\ \epsilon\ ) &= 0 \\
length(\ aS\ ) &= 1 + length(\ S\ )
\end{align*}

\subsection{String Concatenation Specification}
\begin{align*}
s_1 &= \epsilon: & \epsilon \oplus s_2 & &= &s_2 &\\
s_1 &= as'_1: & as'_1 \oplus s_2 & &= &a(s'_1 \oplus s_2)
\end{align*}
\clearpage
\subsection{a)}
Show that if \(s \in S\), then \(s \oplus \epsilon = s\)\\
\textbf{Base Case 1}\\
\[
s \equiv \epsilon
\]
\[
\epsilon \oplus \epsilon = \epsilon
\]
The first base case is when s is structurally equivalent to the empty string \(\epsilon\) and it's a fact that concatenating two empty strings will yield another empty string.

\textbf{Base Case 2}\\
if \(length(s) = 1\) then
\[
s \oplus \epsilon = s
\]

In this second base case we take s to be any string of length 1 and again it is a fact that any string concatenated with the empty string \(\epsilon\) will be the original first string. In this case the string will still be of length 1.

\textbf{Inductive Step}\\
We assume that the base case holds for all lengths of s greater than 1, therefore:

if \(length (s) > 1\)
\[
s \oplus \epsilon = s
\]
\textbf{Q.E.D.}
\subsection{b)}
Show that if \(s_1, s_2 \in S\) then \(length(s_1 \oplus s_2) = length(s_1) + length(s_2)\)\\
\textbf{Base Case 1}\\
\(s_1, s_2 \equiv \epsilon\)

Given that \(length(\epsilon) = 0\) then it goes to say that \(length(\epsilon \oplus \epsilon) = 0\).\\

\textbf{Base Case 2}\\
\(s_1 \neq \epsilon\), where \(length(s_1) = 1\) and \(s_2 \equiv \epsilon\).
\[
length(s_1 \oplus \epsilon) = length(s_1)
\]

For the case where \(s_1\) is a string of length 1 and \(s_2\) is an empty string, the concatenated length will be the length of \(s_1\).\\

\textbf{Base Case 3}\\
\(s_1, s_2 \neq \epsilon\), where \(length(s_1) = 1\) and \(length(s_2) = 1\).
\[
length(s_1 \oplus s_2) = length(s_1) + length(s_2) = 2
\]

For the case where both \(s_1\) and \(s_2\) are strings of length 1, the concatenated length will be the sum of the individual lengths, which is 2.

\textbf{Inductive Step} \\
\(length(s_1) > 1\) and \(length(s_2) > 1\)
\[
length(s_1 \oplus s_2) = length(s_1) + length(s_2)
\]

We assume the base case 3 holds for strings \(s_1, s_2\) of any length greater than 1.\\

\textbf{Q.E.D.}
\section{Problem 2}
\inputminted{prolog}{list_check.pl}
\end{document}
