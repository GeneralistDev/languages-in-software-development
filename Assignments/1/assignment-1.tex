\documentclass[11pt, english, numbers=endperiod]{scrartcl}

\usepackage{listings}
\usepackage{color}

\title{Assignment 1}
\subtitle{COS30023 - Languages in Software Development}
\author{Daniel Parker - 971328X}

\date{\today}

\begin{document}
\maketitle

\section{Proof Tree}


\section{Bubble Sort}
\subsection{bubble/2}
In planning this predicate, I rationalised the result I wanted as being the sorted result of the head of the current list and the head of the tail of the current list. Then I would keep the smaller of the two, and recursively bubble the bigger part as the head of the remaining tail.

\subsection{bubble\_sort/2}
The bubble\_sort predicate makes use of three other predicates; bubble/2, reverse/2, and remove/3. This predicated is again recursive in nature. A sorted list is created by bubbling, reversing the result, removing the top most element (head), reversing it back again and calling the bubble sort on the remaining tail.

\subsubsection{Source}
\lstinputlisting[language=Prolog]{bubblesort.pl}

\section{Logical Circuits}

\end{document}
